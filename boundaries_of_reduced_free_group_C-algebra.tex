\section{Boudnaries of reduced free group C*-algebras (Narutaka Ozawa)}
\cite{ozawa2006boundaries}
Let $\Gamma$ be a free group with rank $n$ ($2 \leq n < infty$).
A measure $\mu$ on $\d \Gamma$ is called quasi-invariant if for any measurable subset $A \subset \d \Gamma$ and any $s \in \Gamma$,
one has $\mu (s A) = 0$ if and only if $\mu (A) = 0$.
A measure $\mu$ on $\d \Gamma$ is called doubly ergodic if the diagonal action of $\Gamma$ on ($\d \Gamma^2,\mu^{\otimes 2}$) is ergodic.
Let a measure $\mu$ on $\Gamma$ be quasi-invariant and doubly-erogodic.

\begin{theorem}
  Under the above condition,
  If
  \begin{align*}
    \theta : C(\d \Gamma) \rtimes_r \Gamma) \rightarrow L^\infty(\d \Gamma, \mu) \rtimes \Gamma
  \end{align*}
  is a completely positive map with $\theta |_{C_r^*(\Gamma)} = \id_{C_r^*(\Gamma)}$, then $\theta = \id$.
\end{theorem}

\begin{corollary}
  $C(\d \Gamma) \rtimes_r \Gamma)$ sits between $C_r^*(\Gamma)$ and its injective envelop $I(C_r^*(\Gamma))$.
\end{corollary}

\begin{proposition}
  If
  \begin{align*}
    \varphi : C(\d \Gamma) \rightarrow L^\infty(\d \Gamma, \mu)
  \end{align*}
  is a unital positive $\Gamma$-equivariant map, then $\varphi = \id$
\end{proposition}

This proposition is key.

\begin{proof}[proof of proposition]
  Any $\Gamma$-equivariant Borel map from $\d \Gamma$ to $\mathcal{M}(\d \Gamma)$ has image in $\mathcal{M}_{\leq 2}(\d \Gamma)$ $\mu^{\otimes 2}$-a.e.

  Fix a dense $\Gamma$-invariant subalgebra $\mathcal{C}$ which is algebraically generated by a countable set.
  There exist a $\Gamma$-equivariant Borel map
  \begin{align*}
    \varphi_* : \d \Gamma \ni \xi \mapsto \varphi_*^\xi \in \mathcal{M}(\d \Gamma)
  \end{align*}
  s.t. for $\mu$-a.e. $\xi \in \d \Gamma$,
  \begin{align*}
    \forall f \in \mathcal{C} \; \int f(\eta) d \varphi_*^\xi(\eta) = \varphi(f)(\xi).
  \end{align*}
  We consider the $\Gamma$-equivariant Borel map
  \begin{align*}
    \d \Gamma^2 \ni (\xi,\eta) \mapsto \varphi_*^\xi + \delta_\xi + \delta_\eta.   
  \end{align*}
  By the first argument, $\varphi_*^\xi = t \delta_\xi + (1-t) \delta_\eta$.
  By $\mu$-invarinat, $\varphi_*^\xi = \delta_\xi$ a.e.
  So, $\varphi = \id$.
\end{proof}

\begin{proof}[proof of theorem]
  There exist a faithful normal conditinal expectation $E : L^\infty(\d \Gamma, \mu) \rtimes \Gamma L^\infty(\d \Gamma, \mu)$.
  We suffices to show that $E \circ \theta|_{C(\d \Gamma)}$ is a unital positive $\Gamma$-equivariant map.
  We note taht $E(s. x) = s. E(x)$.
\end{proof}
